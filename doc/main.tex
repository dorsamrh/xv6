\documentclass{article}


\usepackage[a4paper, margin=0.5in,]{geometry}



\usepackage{anyfontsize}
\usepackage{xcolor}

\usepackage{graphicx}
\usepackage{setspace}
\usepackage{hyperref}
\usepackage{minted}
\usepackage{xepersian}
\hypersetup{colorlinks,urlcolor=blue}
\settextfont[Scale=1.1]  {Vazirmatn-Medium}

	
	
	\begin{document}
		\linespread{3}
		\setstretch{1.5}
		
		\begin{center}
			\includegraphics{logo.png}  
			
			\Large{دانشکده مهندسی کامپیوتر}
			
			\Large{درس سیستم عامل}
			\Large{
			دکتر رضا انتظاری ملکی
			}
			\\
			~\\
			~\\
			\hrule 
			~\\
                                 فاز اول پروژه
                                 		\end{center}
			~\\
			
			\hrule 
			~\\
			~\\
			~\\
			~\\	


			\begin{center}



			
			طراح
			\dotfill 
			مرتضی ملکی نژاد شوشتری


			
			تاریخ انتشار 
			\dotfill 
			۲۷ آبان ۱۴۰۳

			
			تاریخ تحویل
			\dotfill
			۲۳ آذر ۱۴۰۳
						\end{center}
			
			\newpage

		\section{توضیحات}
		\subsection{مقدمه}
		xv6
		یک سیستم عامل آموزشی است  که در درس سیستم عامل از آن استفاده میشود.
		شما در این فاز از پروژه باید بر روی سیستم عامل xv6 قابلیت تعیین بیشینه استفاده از منابع  ( CPU و Ram ) را برای پروسه ها اضافه کنید. به این صورت که اگر پروسه ای از میزان تعیین شده، CPU بیشتری نیاز داشت، نباید بتواند بگیرد و اگر مصرف مموری آن از حد تعیین شده فراتر رفت، دیگر نتواند حافظه ای بگیرد.
		
		\subsection{CPU}
		برای این قسمت، شما باید کدی بزنید که اجازه ندهد در یک ثانیه یک پروسه  بر روی پردازنده بیش از درصد مشخصی در حال اجرا باشد.  برای سادگی حالت تک هسته ای را در نظر بگیرید و فرض کنید که ثانیه ها مجزا از هم هستند (یعنی اگر یک پروسه محدودیت ۵۰ درصد داشته باشد، ممکن است که ۵۰۰ میلی ثانیه آخر ثانیه اول و ۵۰۰ میلی ثانیه اول ثانیه دوم در حال اجرا باشد.)
		
		در واقع شما باید یک syscall برای تعیین محدودیت بنویسید و scheduler موجود در xv6 را به گونه ای تغییر دهید که  به یک پروسه که محدودیت خود را رد کرده  اجازه schedule شدن ندهد.
		
		\subsubsection{
			syscall
			های مورد نیاز
		}
		برای این قسمت باید syscall زیر به xv6 اضافه شوند:
		\begin{minted}[mathescape, linenos]{c}
int set_limit(int limit);
		\end{minted}
		نباید امضای تابع
		\footnote{\lr{Function Signature}}
		باقی syscall ها تغییر کند ولی تغییر بدنه آن ها مشکلی ندارد. 
		(دقت کنید در قسمت بعد این syscall قرار است تغییر کند.)
		
		\subsubsection{
			برنامه های سمت کاربر
		}
		باید برنامه ای بنویسید که سمت کاربر اجرا شود و با استفاده از آن تاثیر محدود کردن \lr{CPU} را نشان دهید. برای اینکه تاثیر مشهود باشد یک کار \lr{CPU intensive}، برای مثال یک حلقه \lr{for} طولانی در نظر بگیرید و آن را چندبار با لیمیت های مختلف اجرا کنید.
		
		\subsection{Ram}
		برای این قسمت باید کد XV6 را به گونه ای تغییر دهید که اگر میزان رم درخواستی یک پروسه از حدی که تعیین شده بیشتر شد، حافظه به آن اختصاص نیابد. برای سادگی فقط حافظه heap را در نظر بگیرید و کافیست کد تابع malloc موجود در فایل umalloc.c را به گونه ای تغییر دهید که میزان حافظه ای که  کنون با استفاده از malloc تخصیص یافته هیچوقت از میزان محدودیت تعیین شده بیشتر نشود. همچنین باید تابع free را نیز بگونه ای تغییر دهید تا حافظه تخصیص داده شده ای که کاربر آزاد کرده، از میزان مصرف آن کم شود و در محدودیت لحاظ نشود. 
		
		\subsubsection{
			syscall
			های مورد نیاز
		}
		سمت کرنل باید syscall قبلی به طوری تغییر کند که امکان تعیین حداکثر RAM باشد، همچنین یک sycal جدید هم باید پیاده شود تا توابع malloc و free که در \lr{User space} در حال اجرا هستند، بتوانند به کرنل اطلاع دهند که میزان مصرف رم کم و یا زیاد شده و اگر لیمیت را رد کرده، خروجی syscall باید $-1 $ باشد.
		
		
		
		\begin{minted}[mathescape, linenos]{c}
int set_limit(int cpu_quota, int memory_quota);
int increase_memory_usage(int amount);
		\end{minted}
		( برای آزاد کردن \lr{ram} یا میتوانید جوری پیاده سازی کنید که تابع \lr{increase\_memory\_usage} ورودی منفی بگیرد و یا یک \lr{syscall} جدید پیاده سازی کنید.)
		
		\subsubsection{برنامه های سمت کاربر}
		باید یک برنامه بنویسید که به تعداد معین شده (مثلا ۱۰) بار حافظه \lr{malloc} کند و در آخر برنامه تعداد \lr{allocation} های موفق و ناموفق را بنویسد. سعی کنید که محدودیت های مختلف و تاثیر آن ها را تست کنید.
		\section{
			 آشنایی با \lr{xv6}
		}
		چالش اصلی شما در این پروژه، آشنایی با \lr{XV6} ، چگونگی کارکرد بخش های مختلف و هدف فایل های مختلف آن است. به همین منظور، میتوانید از لینک های زیر استفاده کنید.
		
		\subsection{منابع}
		\begin{itemize}
			\item 
			اضافه کردن یک اپ سمت کاربر در \lr{XV6}
			\href
			{https://www.geeksforgeeks.org/xv6-operating-system-add-a-user-program}{لینک}

			\item 
			اضافه کردن یک 
			\lr{syscall}
			به 
			\lr{XV6}
			(یه مسئله که تو این لینک گفته نشده، اینه که امضای تابع رو باید تو فایل defs.h هم اضافه کنید.)
			\href{https://www.geeksforgeeks.org/xv6-operating-system-adding-a-new-system-call/}{لینک}
			
			\item
			بخش های مرتبط این کتاب میتونه مفید باشه (مباحثش زیاده ولی قسمت های مرتبط رو بخونید)
			\href{https://pdos.csail.mit.edu/6.828/2023/xv6/book-riscv-rev3.pdf}{لینک}
			
			\item 
			این هم فورک ریپو \lr{XV6} هست که خودم یه مقدار تغییر دادم که یک سری مشکلاتی که موقع compile ممکن هست بخورید رو نداشته باشید. میتونید جهت راحتی از این ریپو به جای ریپو اصلی \lr{XV6} استفاده کنید.
			\href{https://github.com/shoshtari/os-4012-project1}{لینک}
		\end{itemize}

		
		\subsection{نکات}
		\begin{itemize}
	
			\item 
			هدف این پروژه این نیست که یک سیستم عامل بهینه و کارا بزنید،  بلکه این است که یک پیاده سازی عملی از مفاهیمی که در کلاس یاد گرفته اید را ببینید. بنابراین بر روی بهینه بودن کد وسواس به خرج ندهید و هر کدی که نیازمندی های پروژه را برآورده کند کافی است.
			\item 
			محدودیت ها باید به صورت 
			\lr{Hard limit}
			باشند. به این معنا که اگر یک پروسه محدودیت خود را رد کرد، حتی اگر \lr{CPU} کار دیگری نداشته باشد نباید در آن ثانیه آن پروسه را رد کند. دررابطه با RAM صرفا کافیست که تابع malloc در صورتی که نتواند حافظه را اختصاص دهد خروجی ۰ بدهد.
			
			\item 
			در آخرین ورژن های لینوکس، ممکن است نتوانید \lr{XV6} را کامپایل کنید، اگر مشکل داشتید یا از \lr{VM} استفاده کنید یا اگر از ریپو فورک شده استفاده میکنید و داکر را نصب دارید میتوانید با استفاده از دستور زیر پروژه را کامپایل کنید:
			\begin{minted}[mathescape, linenos]{bash}
make docker
			\end{minted}
			
		\end{itemize}
		
		\section{ارزیابی}
	 ارزیابی فاز اول پروژه بر اساس پیاده سازی انجام داده شده و تسلط به مفاهیم و ساختار xv6 است.		
	\end{document}